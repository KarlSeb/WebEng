\documentclass[a4paper]{report}
\usepackage[utf8]{inputenc}
\usepackage[ngerman]{babel}
\usepackage{graphicx}
\usepackage[hyphens]{url}
\usepackage[hidelinks]{hyperref}
\usepackage[toc,nonumberlist,style=altlist]{glossaries}
\usepackage{longtable}
\usepackage{caption}
\usepackage{booktabs}
\usepackage{float}
\usepackage{listings}
\usepackage{newclude}

\newenvironment{code}{\ttfamily}{\par}

\begin{document}
\begin{titlepage}
	\centering
	{\scshape\Huge QA-Project}
	\vspace{1cm} \\
	{\scshape\LARGE Web Engineering - PLS}
	\vspace{1cm} \\
	{\scshape\LARGE Handbuch}
	\vspace{1cm} \\
	{\large\itshape Pascal Süß - 72273} \\
	\vspace{0.5cm}
	{\large\itshape Lukas Wallisch - 73242} \\ 
	\vspace{0.5cm}
	{\large\itshape Sebastian Karl - 72138} \\ 
	\vfill
	{\large \today}
\end{titlepage}
\title{Handbuch}

\tableofcontents

\part{Frontend}

\part{Backend}
\chapter{Einfügen von Testdaten}
\label{admin}
Unabhängig von der im Folgenden beschriebenen Einstellung wird ein Admin-Nutzer angelegt, der als einziger Nutzer die Datenbank-Konsole erreichen kann. Die Anmelde-Daten für diesen Nutzer lauten: Username: \textit{admin}, Passwort: \textit{pass}
\section{Ein- und Ausschalten per Konfigurationsdatei}
Der Server generiert standardmäßig Testdaten beim Start. Dies lässt sich verhindern, indem in der \textit{config.properties}, die in der \textit{PLS-Server-1.0.jar} enthalten ist, das \textit{testDataOn}-Flag auf \textit{false} gestellt wird.\\
Alternativ zu \textit{true} und \textit{false} wird folgendes akzeptiert:
\begin{itemize}
	\item \textit{on} und \textit{off}
	\item \textit{yes} und \textit{no}
\end{itemize}
\section{Eingefügte Daten}
Falls die Testdaten generiert werden, werden folgende Nutzer angelegt:
\begin{itemize}
	\item Username: \textit{bob}, Passwort: \textit{bobspw}
	\item Username: \textit{chloe}, Passwort: \textit{cloespw}
	\item Username: \textit{david}, Passwort: \textit{davidspw}
	\item Username: \textit{fred}, Passwort: \textit{fredspw}
	\item Username: \textit{jack}, Passwort: \textit{jackspw}
	\item Username: \textit{kim}, Passwort: \textit{kimspw}
	\item Username: \textit{michelle}, Passwort: \textit{michellespw}
\end{itemize}
Außerdem werden noch 4 Fragen und Antworten angelegt. Dabei ist eine Frage bereits gelöst, eine besitzt noch keine Antworten und die beiden anderen wurden schon beantwortet, allerdings wurde noch keine Antwort akzeptiert.
\chapter{Start der Applikation}
Zum starten der Applikation muss der Befehl \textit{java -jar PLS-Server-1.0.jar} im Verzeichnis in dem sich die .jar-Datei befindet ausgeführt werden. Der Server belegt dabei den Port 8080.
\chapter{Besonderheiten}
\section{Datenbank-Konsole}
\subsection{Aufrufen der Datenbank-Konsole}
Die Datenbank-Konsole ist unter \textit{http://localhost:8080/h2-console} zu erreichen. Um die Konsole aufrufen zu können, sind Admin-Rechte erforderlich(siehe Kapitel~\ref{admin}).\\ 
Wenn vom Browser bereits ein Cookie eines anderen Benutzers gespeichert wurde, wird die Meldung \textit{"You need Admin rights to access this Page!"} angezeigt und es muss ein Login mit dem Admin-Nutzer auf der Client-Seite durchgeführt werden. Wurde kein Cookie gespeichert, wird das Standard HTTP-Basic Anmeldefenster des Browsers erscheinen, in das die Nutzerdaten eingetragen werden müssen.
\subsection{Einloggen in die Datenbank}
Hat man die Datenbank erreicht, muss folgendes in die entsprechenden Felder eingetragen werden:\\
\\
\begin{code}
	Driver Class: org.h2.Driver\\
	JDBC URL: jdbc:h2:mem:testdb\\
	User Name: sa\\
	Password: \\
\end{code}
\noindent
Danach muss nur noch der \textit{Connect}-Kopf gedrückt werden.
\end{document}